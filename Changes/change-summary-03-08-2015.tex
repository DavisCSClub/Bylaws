\documentclass[11pt]{article}
\usepackage{fullpage}

\title{Summary of Changes}
\date{March 8, 2015}
\author{Davis Computer Science Club Officer Board}

\begin{document}
	
\maketitle
	
\section{Title Page}

Created a title page for aesthetic reasons only.

\section{History}

The ``History" portion of the bylaws is currently written under Article 10 at the end of the bylaws. The changes proposed are to move ``History" to be the second page of the bylaws to show the revision history immediately after the title page.

\section{Table of Contents}

The table of contents currently shares the same page with Article 1. The changes proposed are for the table of contents and the first article to be on separate pages.
	
\section{Article and Section}

Prepended the terms ``Article" and ``Section" to headers. The reason for doing so is to explicitly set a terminology when referencing headers and subheaders in the bylaws.

\section{Name}

The current bylaws states the following: 

\begin{quote}
``The name of this organization shall be the Davis Computer Science Club, hereinafter referred to as the CS Club, Davis CS Club, or the CS Club @ UC Davis."
\end{quote}

\noindent The purpose of the names given after ``hereinafter" are for the purposes are shortening the organization name throughout the bylaws. However, the current bylaws supply three different references which leads to potential inconsistency throughout the bylaws. The changes propose to reference only one name --- DCSC.\\

\noindent Proposed Changes:

\begin{quote}
	``The name of the organization shall be called the Davis Computer Science Club, hereinafter referred to as DCSC."
\end{quote}

\section{Purpose}

Reorganized the bullet points. Removed the last purpose from the current bylaws.

\begin{quote}
``To provide sustainable funding to support all of these purposes."
\end{quote}

\noindent It is to our belief that funding is an implicit necessity of a given organization.

\section{General Meetings}

Current bylaws state the following:

\begin{quote}
	``The Davis CS Club shall conduct regular, open general meetings on every Wednesday, excluding during break, first week, dead day, and finals week, with each meeting convening at 5:30 PM and adjourning after 6:00 PM."
\end{quote}

\noindent Our officer board has decided to run general meetings on the first Wednesday of every month, hence this bylaw conflicts our goals. At the same time, the bylaws are rigid by specifying the times in which the meeting must begin and end.

\noindent The changes proposed are to set the general meetings to be the first Wednesday of every month beginning at the most convenient time of members.

\begin{quote}
``The Davis CS Club shall conduct regular, open general meetings on the first Wednesday of every month, excluding during break, dead day, and finals week. The general meetings shall convene at the time most convenient for a majority of the members and adjourn at the discretion of the officer board."
\end{quote}

\section{Officer Meetings}

Current bylaws state the following:

\begin{quote}
	``The Core Body shall conduct regular Core Officer Meetings on every Monday, excluding during break, first week, dead day, and finals week, with each meeting convening at 5:30 PM and adjourning after 6:00 PM."
\end{quote}

\noindent The date and time specified are rigid in that it has the potential to be at a time in which all board officers are unable to attend due to academic reasons. It is also unreasonable to assume that officer meetings conclude within 30 minutes. The proposed changes remove the constraint of a date and time, but enforces the weekly aspect. 

\begin{quote}
	``The Core Body shall conduct regular Core Officer Meetings every week, excluding during break, dead day, and finals week."
\end{quote}

\section{Officer Eligibility}

Current bylaws require officers to attend three full meetings and one event. There is ambiguity as to what is considered a meeting and what is considered an event. From here on, committee planning meetings and officer meetings are considered meetings while everything aside from review sessions are considered events. The changes proposed change the officer eligibility to three events and one officer meeting.

\section{Election Eligibility Requirements}

In the current bylaws, it is somewhat ambiguous as to whether the officer eligibility section was a reference to existing officers or potential officers. Therefore, an additional section under the elections article was created specifying the eligibility requirements for elections. 


\section{Committee Elections}

In the current bylaws, committee succession is done through the general elections. The changes proposed allows the committee chair and vice-chair to perform an internal election for the committee chair position within the committee. The core officer board may veto this decision based on a simple majority.

\end{document}